\documentclass[9pt,a4paper,sans]{moderncv}        
\moderncvstyle{classic}                             
\moderncvcolor{blue}           
\usepackage[utf8]{inputenc}                  
%\usepackage[applemac]{inputenc}
\usepackage[francais]{babel}
\usepackage[scale=0.89]{geometry}


\name{Boris}{Aïoun}
\title{Spécialisé en informatique, traitement de l'image et du signal}


\address{24 Rue Humbert II }{38000 Grenoble}{France}
\phone[mobile]{+33 6 88 00 81 00}                  
\email{borisaioun@gmail.com}                              
\extrainfo{Age: 26 ans}


\quote{Développeur Logiciel}




\begin{document}
\makecvtitle
\normalsize

\section{Formation et diplôme}
\cventry{2011-2015}{ Diplôme d'ingénieur}{Grenoble-INP Phelma}{}{\textit{Filière Signal Image Communication Multimédia (SICOM). Option Système de Traitement de l'Information et Communications (STIC)}}{}  
\cventry{2009--2011}{Classe Préparatoire Scientifique}{Lycée Blaise Pascal}{Orsay (91)}{\textit{Option: Physique et Sciences de l'ingénieur (PCSI/PSI)}}{}
%\cventry{2009}{Baccalaur\'eat Scientifique}{Lyc\'ee de la Vall\'ee de Chevreuse}{Gif-sur-Yvette (91)}{\textit{sp\'ecialit\'e physique chimie (mention Bien)}}{}


\section{Expériences}

\cventry{Avril 2017 - Actuel}{Concepteur Logiciel }{Sogeti High Tech, Montbonnot (38)}{Développement sur un projet de R\&D de détection de panneau}{Java, Eclipse, OpenCV, Android}{}

\cventry{Novembre 2015 - Mars 2017}{Ingènieur Développeur Logiciel}{Dolphin Integration, Meylan (38)}{Développement et maintien de l'IHM de logiciel de conception et de simulation de circuit électronique}{C, C++, TCL, wxWidget, Visual Studio, SVN, Linux}{}

\cventry{Février-Août 2015}{Stagiaire, Optimisation de Traitement Vidéos}{MBDA, Le Plessis-Robinson (92)}{Etude d'une chaine de traitement vidéo au standard GigEVision, Optimisation d'un algorithme d'incrustation de vidéos en C++ en utilisant OpenCL}{Etude de l'interopérabilité entre OpenCL et OpenGL}{}

\cventry{Mai-Juillet 2014}{Stagiaire, Développement d'un capteur de dents pour vélo électrique}{GIPSA-LAB}{Equipe AGPIG}{Programmation en langage C sur micro-contrôleur STM32 dans le but de détecter les passages de dents du pédalier d'un vélo électrique afin d'en calculer la force développée par le cycliste.}{}

%\cventry{Juin-Août 2012}{Opérateur monitoring des équipements}{STMicroelectronics}{Crolles (38)}{}{Stage Opérateur effectué en salle blanche dans l'équipe Métrologie. }



\section{Langue}
\cventry{Anglais}{Lu/Ecrit/Parlé}{Niveau C1 au BULATS}{Equivalent à environ 945 au TOEIC}{}{}


%\section{Stages Linguistiques}
%\cventry{Eté 2008}{San Raphael (Californie)}{3 semaines en campus universitaire}{}{}{}
%\cventry{Toussaint 2007}{Bedford (Angleterre)}{1 semaine en famille d'accueil}{}{}{}

\section{Competences}
\cvitem{Compétences:}{Informatique , Traitement du signal, Traitement d'images/Vidéo,  Compression d'image/vidéo, Télécommunication, Électronique}
\cvitem{Langages:}{C, C++, Java, TCL, VHDL, Matlab, Assembleur, Altium, Quartus, \LaTeX}
\cvitem{Outils:}  {Visual Studio, Eclipse, Unix, Windows, SVN}
\cvitem{Frameworks: } {wxWidget, OpenCL, OpenCV, OpenGL, CUDA}


\section{Projets}

%\cventry{Binôme}{Programmation d'un Simulateur MIPS}{Langage C}{Septembre-Décembre 2013}{}{}

%\cventry{Binôme}{Conception d'un filtre numérique à réponse impulsionnelle finie et implantation sur FPGA}{apprentissage du flot de conception}{description en VHDL du composant}{Février 2014}{}{}

\cventry{Groupe de 5}{Évaluation de la plateforme Odroid XU en vue d'une utilisation dans un véhicule aérien autonome}{Février-Avril 2014}{programmation d'un GPU en OpenCL}{utilisation d'OpenCV}{} 

%\cventry{Groupe de 5}{Projet de création d'entreprise et création d'activité sur une assistance électrique au pédalage pour vélo}{Septembre 2013-Avril 2014}{élaboration d'un business plan pour la création  d'une entreprise commercialisant une assistance électrique au pédalage pour vélo}{}{}

\cventry{Binôme}{Projet Systèmes Matériels}{Septembre 2014-Janvier 2015}{Implémentation d'algorithmes de traitement du signal sur FPGA et sur processeur GPU}{Implémentation d'un filtre sur FPGA avec un kit de développement FPGA, implémentation d'algorithme de traitement d'image en CUDA}{}



%\section{Vie Associative}
%\cventry{2012-2013}{Membre du Cercle Phelma}{(BDE Phelma)}{membre de la team charg\'ee d'organiser l'arriv\'ee des \'el\`eves de premi\`ere ann\'ee. }{}{}{}
%\cventry{2013}{Membre du Grand Cercle}{(BDE de Grenoble-INP)}{Responsable de la Boutique}{}{}{}



\section{Centres d'intérêt}
\cventry{Sports}{Water Polo}{9 ans}{Club athlétique d'Orsay (91)}{}{}{}
\cventry{Aviation}{Pilotage} {60 h de vol (Cessna 152)}{A\'ero-Club AIR-FRANCE bas\'e \`a Toussus-le-Noble (78)}{}{}










\end{document}


